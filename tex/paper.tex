\documentclass[14pt]{article}
\usepackage[utf8]{inputenc}
\usepackage{anyfontsize}
\usepackage{t1enc}
\usepackage{graphicx}
\usepackage{amsmath}
\usepackage{hyperref}
\usepackage{graphicx}
\usepackage{float}
\usepackage{mathtools}
\usepackage{geometry}
\geometry{letterpaper, portrait, margin=0.85ian}
\usepackage{xcolor}
\definecolor{shadecolor}{RGB}{199,239,247}
\usepackage[version=4]{mhchem}
\usepackage{chemist}
\usepackage{float}
\usepackage{array}
\usepackage[document]{ragged2e}
\usepackage[english]{babel}
\usepackage{csquotes}
\usepackage{listings}
\usepackage[backend=biber, isbn=false, doi=false]{biblatex-chicago}
\usepackage{tikz}
\usepackage{pgfplots}
\pgfplotsset{width=9cm}
\usepackage{amssymb}
\usepackage{blindtext}
\usepackage{multicol}
\usepackage{setspace}
\usepackage{tabularx} % Required for the tabularx environment
\usepackage{booktabs} % Required for nicer horizontal rules
\usepackage{array} % Required for defining custom column types
\usepackage{chemformula}
\usepackage{multirow}

\title{\fontsize{32}{36}\selectfont RC Circuits and Filters}
\author{\fontsize{16}{20}\selectfont Sophie Bergstrom, Gavi Fischer, Youssef El Gharably}
\date{PHY422 - Advanced Methods of Physics and Astrophysics Lab Report}

\newcolumntype{C}{>{\centering\arraybackslash}X}

% Chem Equations
% \ch{CH3COOH <=> CH3COO- + H+}

% Figures 
%\begin{figure} [H]
%    \centering
%    \includegraphics[width=140mm,scale=1]{my_q.png}
%    \caption{$Q$ values plotted logarithmically over the course of 28 days with a rational function fit. The horizontal black dashed line represents my estimation of the equilibrium value, 2.8.}
%    \label{fig:my_label}
%\end{figure}

% Cool Table
%\begin{table}[H]
%    \centering
%    \begin{tabularx}{\textwidth}{CCCCCC} % Adjust the number of 'C's as needed
	%        \toprule
	%        Titration Number & Equivalence Point Volume (mL) & Initial Moles of Acetic Acid (mol) & Initial Molarity of Acetic Acid (M or %mol/L) & Moles at Equivalence (mol) & Molarity at Equivalence (M or mol/L) \\ 
	%        \midrule 
	%        1 & 29.2 & 0.00271 & 0.0201 & $5.48\times 10^{-5}$  & $4.25\times 10^{-4}$ \\
	%        2 & 44.4 & 0.00407 & 0.0282 & $2.74\times 10^{-5}$ & $1.90\times 10^{-4}$ \\
	%        3 & 30.4 & 0.00278 & 0.0213 & $0.91\times 10^{-5}$ & $0.70\times 10^{-4}$ \\
	%        Avg. & 34.7 & 0.00319 & 0.0232 & $3.04\times 10^{-5}$ & $2.28\times 10^{-4}$ \\
	%        STD. (Error) & 8.45 & 0.00077 & 0.00438 & $2.30\times 10^{-5}$ & $1.81\times 10^{-4}$\\
	%        \bottomrule
	%    \end{tabularx}
%    \caption{Table showing concentrations and equivalence points of each titration.}
%    \label{tab:results1}
% \end{table}


\begin{document}
\maketitle
\justifying
\begin{center}
	\section*{Abstract}
\end{center}

\begin{multicols}{2}
	\section{Introduction}
	
	\section{Theory}
	(Gain Analysis, will adjust later)\newline In order to reach the complex form of the gain equation, the following expressions was derived:
	\begin{align}
		\left| \frac{V_{out}}{V_{in}} \right|_{hp} &= \frac{1}{\sqrt{1+(1/\omega RC)^2}} \\
		\left| \frac{V_{out}}{V_{in}} \right|_{lp} &= \frac{1}{\sqrt{1+(\omega RC)^2}}
	\end{align}
	In order to derive equations (1) and (2), the complex form of Ohm's Law had to be manipulated as follow, noting that the derivation branches off into two when we start treating each of the filters as separate voltage dividers:
	\begin{align}
		I = \frac{V}{Z} &= \frac{V}{R+Z_c} \\
		\frac{V_{out,hp}}{V_{in}} &= \frac{R}{R + Z_c} \\
		\frac{V_{out,lp}}{V_{in}} &= \frac{Z_c}{R+Z_c}
	\end{align}
	\indent where $Z_c$ is represented as follow:
	\begin{gather}
		Z_c = \frac{-i}{\omega C}
	\end{gather}
	From this point on, we can substitute $Z_c$ from Eq(6) into Eq(4) and Eq(5) separately to find an expression for each of the two filters.
	\subsection{High-Pass Filter:}
	\begin{align}
		\frac{V_{out}}{V_{in}} &= \frac{R}{R-i/\omega C}\cdot \frac{R+i/\omega c}{R+i\omega c} \\
		&= \frac{R^2 + i (R/\omega C)}{R^2 + 1/(\omega C)^2} \\
		&= \frac{(R\omega c)^2 + i(R\omega c)}{(R\omega c)^2 + 1} \\
		&= \frac{(R\omega c)^2}{(R\omega c)^2 + 1} + i\frac{R\omega c}{(R\omega c)^2+1}
	\end{align}
	Next step is to calculate the modulus of 
	
	
	\section{Results and Analysis}
	
	\section{Conclusion}
\end{multicols}

\end{document}
